\section{Introduction}

NEXMD, from Non-adiabatic EXcited state Molecular Dynamics\cite{malone2020nexmd,nelson2020non}, is a software package designed for simulating the non-radiative electronic relaxation taking place in a molecular system after a photo-excitation. The intricate mechanisms taking place between the excited states manifold underpins processes in nature such as vision\cite{polli2010conical} or photosynthesis\cite{scholes2017using}. A detailed understanding of these mechanisms have also allowed a wide range of applications including but not limited to organic light emitting diodes\cite{salehi2019recent}, photovoltaics\cite{wang2014photodegradation,wolfer2014solution,zhugayevych2015theoretical}, field-effect transistors\cite{sirringhaus2000high,lee2016highly}, sensors\cite{satishkumar2007reversible,maksimov2019genetically,li2019perylene,wilson2008photoactive,singer2022unravelling}, photocatalysts\cite{romero2016organic} and solar cells\cite{bredas2009molecular,schmidt2011moving}. The complex excited state electronic structure arising from strong electronic correlations and low dimensionality\cite{tretiak2002conformational}, combined with delocalized and polarizable $\pi$-electrons are key for the generation of mobile charge carriers\cite{cao1999improved}. Such systems typically undergo an efficient non-radiative relaxation\cite{kasha1950characterization} that can take place through several nonadiabatic pathways leading to overall dissipation of an excess of electronic energy into heat. In this context, many physical processes such as internal conversion\cite{robb2000computational}, energy transfer\cite{jiang2017light}, charge separation\cite{huix2015concurrent,bredas2004charge}, exciton self-trapping\cite{adamska2014self} or vibronic coherences\cite{freixas2020vibronic,keefer2021monitoring,freixas2022ultrafast} can be of relevance.

NEXMD is able to simulate these processes from a full-atom perspective treating molecules that have a few hundreds of atoms, including a few dozens of excited states and for a few picoseconds. The electronic structure is based in the Colective Electronic Oscillator (CEO) method\cite{tretiak2002density}, considering Configuration Interaction Singles\cite{thouless2014quantum} or the Time-Dependent Hartree-Fock (TDHF)\cite{jorgensen1975molecular,mclachlan1964time} combined with semiempirical Hamiltonians models. The initial implementation (NEXMD Version 1.0), included the Trajectory Surface Hopping approach\cite{tully1990molecular}. Recent implementations have added Ehrenfest dynamics\cite{fernandez2016non} and the Ab-Inition Multiple Cloning sampling for Multiconfigurational Ehrenfest\cite{freixas2018ab}. These methods in combination with other features present in NEXMD allow a comprehensive investigation of a wide range of processes and systems, including but not limited to polymers\cite{clark2012femtosecond,oldani2014modeling,ondarse2014computational,alfonso2016interference,sifain2018photoexcited,ondarse2018let,mukazhanova2023impact}, dendrimers\cite{freixas2022ultrafast,ondarse2016ultrafast,galindo2015dynamics,fernandez2012shishiodoshi,soler2012analysis,ondarse2018energy,freixas2019photoinduced,bonilla2023impact}, nanorings and nanobelts\cite{franklin2017phonon,oldani2017photoinduced,rodriguez2018modification,franklin2016carbon,freixas2022infinitene,negrin2023photoexcited}, light harvesting complexes\cite{bricker2015non,shenai2015internal,zheng2017photoinduced} and energetic materials\cite{greenfield2015photoactive,nelson2016ultrafast,lystrom2018site}.

This document is a user manual covering technical details required for using NEXMD, such as available features, compilation and execution, input file extensive description and output files formatting. It also contains a section with input files examples intended to be used as templates for the specific features available within NEXMD. This manual is not intended to cover the theoretical background supporting the features of NEXMD. The interested reader can use instead a recent extensive review on NEXMD\cite{nelson2020non} or specialized papers on the development of a given feature or application. While this manual is a useful guide for the new user, more guided tutorials (specific to particular use cases) are available elsewhere.