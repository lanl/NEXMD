\section{Restarting Simulations}
\label{restart}
Non-adiabatic trajectories may not finish within the user-defined number of classical steps for several reasons such as (1) the computing system may have a time-limit that is less than the time to complete a trajectory, (2) the defined wall-time may not be long enough to complete the trajectory, or (3) a problem in the computing system may cause jobs to stop before completion.  In any case, trajectories may be restarted from the last time-step, as long as the last excited-state, nuclear coordinates and velocities, and quantum coefficients are known.  These quantities are available in the file \verb+restart.out+, which has exactly the same structure as an \verb+input.ceon+ but with the data corresponding to the restarting point. The file \verb+restart.out+ is created automatically by NEXMD. If it is corrupted for some reason, for instance if the program stopped exactly while writing it and is incomplete, it might be reconstructed from the last step. For restarting we can launch NEXMD exactly as we did the first time. NEXMD will look for a \verb+restart.out+ file and, if it is there, NEXMD will read data from it and continue from where it stopped. Any exceeding lines in the output files, if any, will be deleted. For the simulation to start from the beginning there can't be any file with the name \verb+restart.out+ (or \verb+restart_????.out+ for AIMC) in the directory.

It is also important to use a three spaces indentation in the \verb+input.ceon+ file for the variables in the \verb+restart.out+ file to be updated.

If for some reason the user prefers restarting trajectories from a step different from the one stored in the \verb+restart.out+ file, it can be done by editing the \verb+restart.out+ file and setting the corresponding initial time, simulation steps, coordinates, velocities and electronic coefficients. This should be done with caution because NEXMD will remove any exceeding lines from the output files when restarting. For AIMC dynamics, all restarts should be synchronized.

If any of the output files is corrupted, i.e. it has less lines than what it should for restarting at time set in the input, the simulation will stopped ant the corresponding corrupted file will be notified in the \verb+md.out+ file.